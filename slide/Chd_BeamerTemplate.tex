\documentclass[10pt,aspectratio=169,mathserif]{beamer}		
%设置为 Beamer 文档类型,设置字体为 10pt,长宽比为16:9,数学字体为 serif 风格

%%%%-----导入宏包-----%%%%
\usepackage{ccnu}			%导入 CCNU 模板宏包
%\usepackage{ctex}			 %导入 ctex 宏包,添加中文支持
\usepackage{xeCJK}
\usepackage{amsmath,amsfonts,amssymb,bm}   %导入数学公式所需宏包
\usepackage{color}			 %字体颜色支持
\usepackage{graphicx,hyperref,url}	
%%%%%%%%%%%%%%%%%%


\setsansfont{Helvetica} 							%Windows和Mac OS下都可用
%\setsansfont{Times New Roman}
%\setCJKmainfont{Hiragino Sans GB W3}    	%仅Windows可用
\setCJKmainfont{Songti SC}								%仅Mac OS下可用

% \beamertemplateballitem		%设置 Beamer 主题
\usetheme{Copenhagen}
\usecolortheme{orchid}


\AtBeginSection[]
{
  \begin{frame}<beamer>
    \frametitle{\textbf{目录}}
    \textbf{\tableofcontents[currentsection]}
  \end{frame}
}


%%%%----首页信息设置----%%%%
\title[Full-domain Apparent Resistivity Definition with Arbitrary Transmitting Waveform]{\fontsize{13pt}{18pt}\selectfont {Full-domain Apparent Resistivity Definition for Loop Source TEM with Arbitrary Transmitting Waveform}}
%\subtitle{\fontsize{9pt}{14pt}\selectfont \textbf{子标题}}			
%%%%----标题设置


\author[Xu Jing]{Xu Jing \\ \texttt{jx\_tdem@qq.com}}
%%%%----个人信息设置

\institute[Group TDEM of Prof. Xiu Li]{
  College of Geology Engineering and Geomatics\\
  Chang’an University}
%%%%----机构信息

\date[\today]{
 \today}
%%%%----日期信息

%=====================================================
\begin{document}

\begin{frame}
\titlepage
\end{frame}				%生成标题页
%=====================================================


\section*{目录}

	\begin{frame}
		\frametitle{\textbf{目录}}
		\textbf{\tableofcontents}
	\end{frame}
%=====================================================


\section{引言}

	\begin{frame}
		\frametitle{\textbf{引言}}
		\begin{block}{\textbf{研究背景}}
			\begin{itemize}
				\item 研究背景一
				\item 研究背景二
				\item 研究背景三
			\end{itemize}
		\end{block}
		
		\begin{block}{\textbf{研究现状}}
			\begin{itemize}
				\item 研究现状一
				\item 研究现状二
				\item 研究现状三
			\end{itemize}
		\end{block}
	\end{frame}
%=====================================================


\section[标题简写]{标题的完整表达}

	\begin{frame}
		\frametitle{\textbf{二级标题1}}
		\begin{figure}[!t]
			\centering
			\includegraphics[width=1in]{figures/B_logo.jpg}
			\caption{长大蓝色Logo-这里填图名}
			\label{figure1}
		\end{figure}
		\textbf{粗体文字}普通文字。\\
		换行书写。
	\end{frame}
	
	\begin{frame}
		\frametitle{\textbf{二级标题2}}
		\begin{figure}[!t]
			\centering
			\includegraphics[width=4in]{figures/D_logo.jpg}
			\caption{长大黑色文字Logo-这里填图名}
			\label{figure2}
		\end{figure}
		\textbf{粗体文字}普通文字。\\
		换行书写。
	\end{frame}
	
	\begin{frame}
		\frametitle{\textbf{二级标题3}}
		\begin{figure}[!t]
			\centering
			\includegraphics[width=1in]{figures/B_logo.jpg}
			\caption{长大蓝色Logo-这里填图名}
			\label{figure3}
		\end{figure}
		\textbf{粗体文字}普通文字。\\
		换行书写。
	\end{frame}
%=====================================================
\section[标题三]{标题三的完整表达}

	\begin{frame}
		\frametitle{\textbf{标题3.1}}
		% ----------------分栏的结构开始---------------- %
		% 该结构中使用block分开两个内容区
		\begin{columns}
			\column{.5\textwidth}%分栏占总宽一半
			\footnotesize
			\begin{itemize}
				\item 第一点
				\item 第二点
				\item 第三点
			\end{itemize}
			
			\column{.5\textwidth}%分栏占总宽一半
			\begin{table}%数据表
				\caption{表名}
				\label{table1}
				\centering
				\footnotesize
				\begin{tabular}{|c|c|}%数据表分两列
					\hline
					\textbf{表头栏1}           & \textbf{表头栏2}\\
					\hline
					内容1    				     &1\\
					\hline
					内容2                      &数据2\\
					\hline
					内容3                      &3\\
					\hline
					内容4                      &4\\
					\hline
					内容5                      &5\\
					\hline
					内容6                      &6\\
					\hline
				\end{tabular}
			\end{table}
		\end{columns}
		
	\end{frame}

	\begin{frame}
		\frametitle{\textbf{标题3.2}}
		\begin{figure}[!t]
			\centering
			\includegraphics[width=4in]{figures/D_logo.jpg}
			\caption{长大黑色文字Logo-这里填图名}
			\label{figure4}
		\end{figure}
		\textbf{粗体文字}普通文字。\\
		换行书写。
	\end{frame}



\end{document} 